\section{Segítség, építkezem!}

El szeretnénk indítani a \texttt{Segítség, építkezem!} weboldalt, amely a családi
ház építésébe, felújításába kezdő embereket segíti a tervezés és
kivitelezés folyamatában, és az alábbi főbb funkciókkal bír:

\begin{itemize}
    \item Szakemberek és termékek keresése
    \item Tudásbázis
    \item Ütemezést segítő alkalmazás
\end{itemize}

Az alkalmazás használata mind építői, mind kivitelezői, értékesítői oldalról regisztrációhoz kötött!

\section{Szakemberek és termékek keresése}

A specifikáció az alábbi szempontok alapján készült:

\begin{enumerate}
    \item A weboldalon különféle szakemberek és építést/felújítást tervező magánszemélyek regisztrálhatnak.
    \item Szakember keresésének lehetősége megadott paraméterekre leszűrve az alábbi fő kategóriákban:
    \begin{itemize}
        \item Tervezők
        \item Kivitelezők
        \item Műszaki ellenőr
    \end{itemize}
    \item Termékek keresése a tudásbázisban szereplő technológiákhoz kapcsolódóan
    \item Ajánlási rendszer, referenciák kezelése mind szakemberek, mind termékek esetében
    \item Aktív részvétellel kedvezmények szerzése
    \item A specifikációt alkalmazási példa ismertetésével tegye teljessé!
\end{enumerate}

\section{Definíciók}
\subsection{Követelmények azonosítása}
A követelmények egyedi azonosítására az \kovAzon{A\_ID} formátumot használjuk, ahol "A" a követelmények részkategóriáját azonosítja, az "ID" pedig az azon belüli követelmény azonosító.
A követelmények részkategóriái:
\begin{itemize}
    \item R: Regisztráció követelményei
    \item Ka: Kategóriák és keresés követelményei
    \item Sz: Szakemberek keresése
    \item T: Termékek és tudásbázis követelményei
    \item A: Ajánlási rendszer és referenciák követelményei
    \item Ke: Kedvezmények követelményei
    \item E: Egyéb követelmények
\end{itemize}

\subsection{Típusok}
Háromféle típust különböztetünk meg a “Segítség, építkezem!” weboldallal kapcsolatban:
\begin{enumerate}
    \item Szám típus, amely általában egész számokat tartalmaz, de tartalmazhat valós számokat is.
    \item Szöveg típus, amely ritka, viszonylag egyedi szöveges értéket tartalmaz, például a szakemberek esetében egy szakember nevét.
    \item Felsorolás típus, amely magát az adatot szövegesen tárolja, viszont ez a típus már gyakran ismétlődő adatokat reprezentál, például a szakemberek esetében a fő- és alkategóriák megnevezéseit.
\end{enumerate}

Ezeken kívül lehetnek más típusok is, például kép típus a szakemberek esetében, de ezek a típusok csakis tárolásra, valamint megjelenítésre fognak kerülni a weboldalon, semmi más funkcionalitás nincs elvárva ezekkel a megnevezetlen típusokkal kapcsolatban.


\subsection{További definíciók}
\textbf{Szakember kereső:} a “Segítség, építkezem!” weboldalon belül az a weblap, amely a szakemberek keresését teszi lehetővé.\\
\textbf{Termék kereső:} a “Segítség, építkezem!” weboldalon belül az a weblap, amely a termékek keresését teszi lehetővé.\\
\textbf{Alapértelmezett érték:} ha egy értéket a felhasználó szolgáltatna a weboldalnak, de a felhasználó még nem tudta, vagy még nem akarta ezt az értéket szolgáltatni, akkor a weboldal a megnevezett alapértelmezett értéket használja a funkcionalitásaiban, amíg a felhasználó nem szolgáltat más értéket.
