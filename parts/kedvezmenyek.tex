\section{Kedvezmények}

Amennyiben az ügyfél hozzájárul a számára elvégzett munka
referenciának történő felhasználásához különböző mértékű árengedményekhez juthat.

\subsection{Kedvezményszintek}
A kedvezmények különböző szintekbe sorolhatók. Minden szinten a kivetelező megadhatja
a kedvezmény méretét (százalékban) az alkalmaás által megszabott intervallumon belül (pl. 5 - 10\%). 
A kedvezmény ezen korlátozása segíti az ügyfelek bármiféle átverésének, vagy félrevezetésének elkerülését a
szakemberek álltal, akár legyen ez túl alacsony kedvezmény egy részletes referenciáért,
vagy félrevezetően túl nagy kedvezmények, ahol valójában a kezdeti ár túl magas és nem tükrözi a valós piaci viszonyokat.

Az összes kedvezményi szintre vonatkozik, hogy az ügyfél előzetesen hozzájárul a vállalt feltételek teljesítésére, azzal a megkötéssel, hogy
minden médiaanyagot mely a kivitelezéssel kapcsolatban készült, kizárólag az oldalon történő üzletszerzés céljából használhatnak fel a szakemberek
munkájuk minőségének bizonyítására.

\subsubsection{Alapszint (0 - 5\%)} 
Az ügyfél hozzájárulásával lehetőséget biztosít a kivitelező számára, hogy a kész munkáról képeket készítsen. Ez a referencia segíti a kivitelezőt a szakmai portfóliója bővítésében és a leendő ügyfelek tájékoztatásában.
A hozzájárulásért cserébe az ügyfél a kivitelezö álltal előre meghatározott mértékü kedvezményben részesül, amely a szolgáltatás árából kerül levonásra.
\subsubsection{Középszint (5 - 10\%)}
Az alapszinten megfogalmazottak mellett az ügyfél ahhoz is hozzájárul, hogy a kész munkáról videók készüljenek. A videók lehetővé teszik a munkafolyamatok vagy az eredmény még részletesebb bemutatását, segítve ezzel a kivitelező szakmai portfóliójának további erősítését.
A videók készítéséhez való hozzájárulásért az ügyfél magasabb szintű kedvezményben részesül.

\subsubsection{Legfelsőbb szint (10 - 20\%)}
Az előző pontokban megfogalmazottak mellett az ügyfél ahhoz is hozzájárul, hogy a kivitelező személyes bemutatót tarthasson a kész munkáról, előre egyeztetett időpontokban. Ezek a bemutatók kizárólag a referencia célját szolgálják, és segítik a kivitelezőt abban, hogy az elvégzett munkát potenciális ügyfelek számára a helyszínen is bemutathassa.

A bemutatók lebonyolítására egy felső időkorlát kerül meghatározásra, amely legfeljebb 8 bemutatóórát tesz lehetővé, legfeljebb hat hónap időtartamra. Az ügyfél vállalja, hogy legalább 3 alkalommal biztosítja a bemutatóhoz szükséges lehetőséget, havonta legalább egy lehetséges idöpontot garantálva. A bemutatókért cserébe az ügyfél a legmagasabb szintű kedvezményben részesül, amely jelentős mértékben csökkenti a szolgáltatás végső árát. 

\subsection{A kedvezmény alkalmazásának menete}
Amennyiben az adott kivitelező munkájához megadja az árengedmény lehetőségét és az azokhoz szükséges vállalásokat,
az ügyfél a webes felületen dönthet, hogy él-e a lehetőséggel és kiválaszthatja annak szintjét.
Ezt követően a kedvezmény mérete és az ezért cserébe vállalt feltételek a szerződésben rögzítésre kerülnek.

\subsection{A kedvezményrendszer használatának ösztönzése}
Bár a kedvezményrendszerbe való csatlakozás nem kötelezö a kivitelezők számára, errre a saját referenciájuk felállítása mellett más ösztönző tényezők is szóba jönnek:
\begin{itemize}
\item Sok referencia esetén magasabb prioritás kereséseknél
\item Engedmény az oldal használatáért fizetendő közvetítői díjból
\item Kiemelt partnerré válás lehetősége, ajánlás a webalkalmazás kezdőlapján. 
\end{itemize} 